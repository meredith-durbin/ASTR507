\documentclass[11pt]{article}
\usepackage[margin=1in, lmargin=0.75in]{geometry}
\usepackage{caption}
\usepackage{float}
\usepackage{graphicx}
\usepackage{latexsym}
\usepackage{amsmath}
\usepackage{cancel}
\usepackage{astro}
\usepackage{url}
\usepackage[bottom]{footmisc}

\newcommand\allbold[1]{{\boldmath\textbf{#1}}}
\newcommand\pc{\mathrm{\ pc}}
\newcommand\Lpc{\ L_\odot/\!\!\pc^2}
\newcommand\sech{\ \!\mathrm{sech}}

\begin{document}

\begin{flushright}Meredith Durbin\\
Tom Quinn\\
Astro 507: Thermodynamics\\
\today\\

\end{flushright}

\center{\textsc{Homework 5}} \\[6pt]

\begin{enumerate}

\item
	\begin{enumerate}
    \item The thermal de Broglie wavelength of a particle is $\lambda = h/(2\pi mkT)^{1/2}$; for an electron at the center of the sun, $\lambda_e = 1.924573 \times 10^{-9}$ cm. This gives a degeneracy parameter of $n_e\lambda_e^3 = 0.713$.
    
    \item Assuming the sun is entirely hydrogen and disregarding all states other than the ground state, we can use a simplified form of the Saha equation:
    \begin{align}
    \frac{n_p}{n_{HI}} &= n_e^{-1} \lambda_e^{-3} e^{-\chi_I/kT} \\
    &= 1.39
    \end{align}
    Substituting $X = n_p/(n_p + n_{HI}) = n_p/n_H$, we find:
    \begin{align}
    \frac{X^2}{1 - X} &= 1.39 \\
    X &= 0.67
    \end{align}
    \item Assuming that this question refers to ionized hydrogen, we can apply the concept of a Wigner-Seitz cell and find $r_p$ using $n_p = n_e$:
    \begin{align}
    r_p &= \left( \frac{3}{4\pi n_p} \right)^{1/3} \\
    &= 1.34 \times 10^{-9} \mathrm{\ cm} \\
    &= 0.13 \mathrm{\ \AA}
    \end{align}
    The calculation changes only slightly for non-ionized hydrogen, given that $n_H = n_p/0.67$.
    \item The Saha equation is only valid for relatively low-density gases, where $n_e\lambda_e^3 \ll 1$; we found that this was not so in part (a). One cannot treat atoms at the center of the sun in isolation because the mean separation is well under a Bohr radius, which means that the ionization energy is impacted by degeneracy. I would expect that the hydrogen at the center of the sun would be fully ionized.
    \end{enumerate}

\end{enumerate}
\end{document}